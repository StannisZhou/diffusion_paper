


For the energy function, we hand-designed two different kinds of landscape: random well energy, which we use for the region around target $A$, and random crater energy, which we use for the region around target $B$. The basic components of these energy functions are the well component, given by
\begin{equation}
F_w(x|d_w, r) = -\frac{d_w}{r^4}(||x - x_A||_2^4 - 2r^2||x - x_A||_2^2) - d_w
\end{equation}
where $d_w$ gives the depth of the well; the crater component, given by
\begin{equation}
F_c(x|d_c, h, r) = \frac{d_c}{3b^2r^4 - r^6}(2||x - x_B||_2^2 - 3(b^2 + r^2)||x - x_B||_2^4 + 6b^2r^2||x - x_B||_2^2) - d_c
\end{equation}
where $d_c$ and $h$ give the depth and the height of the crater, respectively, and
\begin{equation}
b^2 = -\frac{1}{3d}(-3 d_c r^2 + C + \frac{\Delta_0}{C})
\end{equation}
with
\begin{align*}
C &= 3r^2 \sqrt[3]{d_c h (d_c + \sqrt{d_c (d_c + h)})} \\
\Delta_0 &= -9 d_c h r^4
\end{align*}
and finally a random component, given by
\begin{equation}
F_r(x|\mu, \sigma) = \sum_{i=1}^m\prod_{j=1}^d exp(-\frac{(x_j - \mu_{i j})^2}{2\sigma_{i j}^2})
\end{equation}
where $\mu=(\mu_{i j})_{m \times d}$ and $\sigma=(\sigma_{i j})_{m \times d}$, with $\mu_i=(\mu_{i 1}, \cdots, \mu_{i, d}), i=1, \cdots, m$ being the locations of $m$ Gaussian random bumps in the region around the targets, and $\sigma_{i j}, i=1, \cdots, m, j=1, \cdots, d$ gives the corresponding standard deviations.

To make sure the energy function is continuous, and the different components of the energy function are balanced, we introduce a mollifier, given by
\begin{equation}
F_m(x|x_0, r) = exp(-\frac{r}{r - ||x - x_0||_2^{20}})
\end{equation}
where $x_0=x_A, r=d_A$ or $x_0=x_B, r=d_B$, depending on which target we are working with, and a rescaling of the random componet, which is given by $0.1 * d_w$ if we are perturbing the well component, and $0.1 * (d_c + h)$ if we are perturbing the crater component.

Intuitively, for the well component, we use a $4$th order polynomial to get a well-like energy landscape around the target that is continuous and differentiable at the boundary. Similarly, for the crater component, we use a $6$th order polynomial to get a crater-like energy landscape around the target that is also continuous and differentiable at the boundary. For the random component, we are essentially placing a number of Gaussian bumps around the target. And for the mollifier, we are designing the function such that it's almost exactly 1 around the target, until it comes to the outer boundary, when it transitions smoothly and swiftly to 0.
To summarize, given parameters $d_w, d_c, h$ and random bumps $\mu_{A}, \mu_{B}$ with $\mu^{A}_i\in \dot{A} \setminus A, i=1, \cdots, m_A, \mu^{B}_i \in \dot{B} \setminus B, i=1, \cdots, m_B$, and the corresponding standard deviations $\sigma^{A}, \sigma^{B}$ with $\sigma^{A}_{i j}, i=1,\cdots, m_A, j=1, \cdots, d, \sigma^{B}_{i j}, i=1, \cdots, m_B, j=1, \cdots, d$, the energy function we used in the experiments is given by
\begin{align}
F(x) &= F_w(x|d_w, d_A) + 0.1 \times d_w \times F_m(x|x_A, d_A) + F_r(x|\mu^{A}, \sigma^{A}), \forall x \in \dot{A} \setminus A \\
F(x) &= F_c(x|d_c, h, d_B) + 0.1 \times (d_c + h) \times F_m(x|x_B, d_B) + F_r(x|\mu^{B}, \sigma^{B}), \forall x \in \dot{B} \setminus B
\end{align}

In our actual experiments, we used
\begin{equation*}
d_w = 10.0, 
d_c = 6.0,
h = 1.0,
\sigma^{A}_{i j} = \sigma^{B}_{k, l} = 0.01, \forall i, j, k, l
\end{equation*} and
\begin{equation*}
\mu^{A} = \begin{pmatrix}%
0.512&0.597&0.009&-0.018&-0.009\\%
0.496&0.597&0.009&0.026&-0.013\\%
0.464&0.58&-0.018&-0.007&-0.012\\%
0.528&0.591&-0.029&0.008&0.022\\%
0.469&0.601&-0.022&-0.002&0.014\\%
0.517&0.608&0.013&-0.02&0.021\\%
0.474&0.588&0.005&0.002&0.003\\%
0.532&0.575&0.018&0.005&0.01\\%
0.466&0.6&0.012&0.027&-0.004\\%
0.497&0.617&-0.006&0.019&-0.033%
\end{pmatrix}, 
\mu^{B} = \begin{pmatrix}%
-0.749&-0.019&-0.022&0.014&0.032\\%
-0.701&0.013&0.038&0.02&0.005\\%
-0.683&-0.021&-0.024&-0.02&0.05\\%
-0.756&-0.012&0.016&-0.001&0.013\\%
-0.696&0.008&0.038&0.031&-0.037\\%
-0.68&0.023&-0.041&-0.001&-0.017\\%
-0.73&-0.022&0.031&-0.033&0.011\\%
-0.763&0.008&0.013&0.016&0.004\\%
-0.669&-0.002&-0.006&0.004&-0.033\\%
-0.709&-0.012&0.032&0.024&-0.02%
\end{pmatrix}
\end{equation*}
